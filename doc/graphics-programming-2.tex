\iffalse %% true for article, false for presentation
\documentclass[12pt]{article}
\usepackage{beamerarticle}
\usepackage[lmargin=1.3in,rmargin=1.3in,tmargin=0.5in,bmargin=0.5in,marginparwidth=1.5in]{geometry}
\else
\documentclass[ignorenonframetext]{beamer}
\fi
\usepackage[export]{adjustbox}
\usepackage{mathptmx}
\usepackage{natbib}
\usepackage{apalike}
\usepackage{multicol}
\usepackage{eso-pic}
\usepackage[T1]{fontenc}
\usepackage{tikz}
\usepackage[english]{babel}
\usepackage[latin1]{inputenc}
\usepackage{helvet}
\usepackage{graphicx}
\usepackage{xcolor}
\usepackage{amsmath}
\usepackage{enumitem}

\makeatletter
\gdef\SetFigFont#1#2#3#4#5{%
  \reset@font\fontsize{#1}{#2pt}%
  \fontfamily{#3}\fontseries{#4}\fontshape{#5}%
  \selectfont}%
\makeatother

\setlist[itemize]{itemsep=20pt,label=$\bullet$}

\usepackage{tikz}

\defbeamertemplate<article>{frame begin}{lined}{\par\noindent\rule{\textwidth}{1pt}\par}
\defbeamertemplate<article>{frame end}{lined}{\par\noindent\rule{\textwidth}{1pt}\par}

\newcounter{framebox}
\defbeamertemplate<article>{frame begin}{tikzed}{%
  \par\noindent\stepcounter{framebox}%
  \tikz[remember picture,overlay] %
  \path (-1ex,0) coordinate (frame top \the\value{framebox});}
\defbeamertemplate<article>{frame end}{tikzed}{%
  \hspace*{\fill}\tikz[remember picture,overlay] %
  \draw (frame top \the\value{framebox}) rectangle (1ex,0);\par}


\mode<article>
{
  \setlength{\parskip}{10pt}
  \setlength{\parindent}{0pt}

  \setbeamertemplate{frame begin}[lined]
  \setbeamertemplate{frame end}[lined]
%  \addtobeamertemplate{frame begin}{}{\setbox0=\bgroup}
%  \addtobeamertemplate{frame end}{\egroup}{}
  \usepackage{fancyhdr}
  \renewcommand{\headrulewidth}{0pt}
  \fancypagestyle{plain}{%
    \fancyhead[L]{}
    \fancyhead[C]{}
    \fancyhead[R]{}
    \fancyfoot[C]{}
    \fancyfoot[R]{\raisebox{6in}[0pt][0pt]{\hbox to 0pt{\hspace{0.3in}\framebox{\LARGE\thepage}}}}}
  \pagestyle{plain}
}

\newcommand{\cp}[1]{\textcolor{lightgray}{#1}}
\setcitestyle{authoryear,open={(},close={)}}

\mode<presentation>
{
%  \hypersetup{colorlinks=true,citecolor=green}

  \usecolortheme{default}
  \useinnertheme[shadow]{rounded}
  \useoutertheme{tsinfolines}

  \usesubitemizeitemtemplate{%
    \tiny\raise1.5pt\hbox{\color{beamerstructure}$\blacktriangleright$}%
  }
  \usesubsubitemizeitemtemplate{%
    \tiny\raise1.5pt\hbox{\color{beamerstructure}$\bigstar$}%
  }

  \setbeamersize{text margin left=1em,text margin right=1em}

% Turns off headline.
%  \setbeamertemplate{headline}[default]

  %% started as:  \usetheme[secheader]{Boadilla}
%  \usecolortheme{albatross}
  % or ...

  \setbeamercovered{invisible}
  % or whatever (possibly just delete it)

%  \addtobeamertemplate{frame title}{\begin{center}}{\end{center}}
%  Doesn't seem to do what I'd want.

}

%\beamertemplatenavigationsymbolsempty


\title{VR Graphics Programming for, well, you know}
\subtitle{Some (relatively) easy steps to virtual reality}
\author{Tom Sgouros}
\institute{Center for Computation
    and Visualization\\Brown University\\thomas\_sgouros@brown.edu}
\date{Spring 2018}


% If you wish to uncover everything in a step-wise fashion, uncomment
% the following command:

%\beamerdefaultoverlayspecification{<+->}

\begin{document}

\maketitle

\begin{frame}
\titlepage
\end{frame}

\section{Structure of programs}

\begin{frame}{Program structure}

\begin{center}
\begin{minipage}{0.5\columnwidth}
\begin{itemize}
\item Render loop
\item Event handler
\item Shaders
\end{itemize}
\end{minipage}
\end{center}
\end{frame}

\subsection{Render loop}

\begin{frame}[fragile]{Render loop}
\begin{center}
\begin{minipage}{0.5\columnwidth}

\hfill{\bfseries Initialize stuff}\hfill~

\hfill\hbox{$\downarrow$ ~~~~~~~}\hfill~

\hfill\hbox{\bfseries What changed?}\hfill~

\hfill\hbox{$\downarrow$ ~~~~~ $\uparrow$}\hfill~

\hfill\hbox{\bfseries Draw it}\hfill~
\end{minipage}
\end{center}


\begin{verbatim}
  ...
  scene.addObject(axesSet);

  scene.setLookAtPosition(glm::vec3(0.0f, 0.0f, 0.0f));
  scene.setCameraPosition(glm::vec3(1.0f, 2.0f, 7.5f));

  scene.prepare();

  glutMainLoop();
  return(0);
\end{verbatim}
\end{frame}

\begin{frame}[fragile]{Render function}

Just draws stuff.

\begin{verbatim}

void renderScene() {
  ...
  glm::vec3 pos = tetrahedron->getPosition();
  oscillator += oscillationStep;
  pos.x = sin(oscillator);
  pos.y = 1.0f - cos(oscillator);
  tetrahedron->setPosition(pos);

  glClear(GL_COLOR_BUFFER_BIT | GL_DEPTH_BUFFER_BIT);
  scene.load();
  scene.draw(scene.getViewMatrix(), scene.getProjMatrix());

  glutSwapBuffers();
\end{verbatim}
\end{frame}


\subsection{Event handler}

\begin{frame}[fragile]{Event handler}

Handles asynchronous input!

\begin{verbatim}
void processNormalKeys(unsigned char key, int x, int y) {

  float step = 0.5f;

  switch (key) {
  case 27:    exit(0);
  case 'a':
    scene.addToCameraPosition(glm::vec3(-step, 0.0f, 0.0f));
    break;
  case 'q':
    scene.addToCameraPosition(glm::vec3( step, 0.0f, 0.0f));
    break;
  case 's':
    ...
\end{verbatim}
\end{frame}


\subsection{Shaders}

\begin{frame}[fragile]{Shaders}
OpenGL innovation, historically related to the transformation of
graphics cards to GPUs.  Looks C-ish.  Biggest problem is the lack of
clarity about inputs and outputs.

\begin{verbatim}
#version 120

uniform mat4 projMatrix, viewMatrix, modelMatrix;
attribute vec4 position, color;
varying vec4 colorFrag;

void main()
{
  colorFrag = color;
  gl_Position = projMatrix * viewMatrix * modelMatrix * position ;
}
\end{verbatim}
\end{frame}

\section{Virtual reality}

\subsection{HMD}

\subsection{CAVE}

\subsection{Phones}

\begin{frame}{All about the timing}

\end{frame}


\end{document}
